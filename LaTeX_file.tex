\documentclass{article}
\usepackage[T2A]{fontenc}
\usepackage[utf8x]{inputenc}
\usepackage[russian]{babel}
\usepackage[normalem]{ulem}
\usepackage{amsmath}
\usepackage{tikz}
\usepackage{pgfplots}
\pgfplotsset{compat=1.9}
\textheight=24cm
\textwidth=17cm
\oddsidemargin=-30pt
\topmargin=-2cm
\parskip=5pt
\begin{document}
\begin{center}
{\bf MIPT 2020 presents}



 \begin{Huge}
{\bf <<Bad mathematitian taking deprivatives>>}
 \end{Huge} 

$\copyright$ Annnna Savchuk
\end{center}
{\bf Introduction}

My name is Anna Savchuk and in this paper I will present to caring reader the process of artificial intelligence training to calculate deprivatives. The problems of calculating difficult expression is widely spread specially among \sout{us} first year students, so as their representative I decided to make a program that will be able to differentiate and easily simplify mathematical expressions.

Differentiating complex expressions I reffer to MIPT lections and textbooks but nothing can save me from making mistakes so I must apologize for making some simple calculations. In my defense I can say that they train attention and write derivatives of simple functions into the subconscious. In my work I'll break down a big expression, simplify it, take a derivative and look at it. This process can go on endlessly, but even this is a great success for me.

\begin{flushleft}
{\bf We are going to work with this expression, but before we'll simplify it:}
\end{flushleft}
$$sinx$$
\begin{flushleft}
{\bf So it looks like this:}
\end{flushleft}
\begin{center}\begin{tikzpicture}
\begin{axis}
\addplot coordinates {
(1, 0.841471)
(1.1, 0.891207)
(1.2, 0.932039)
(1.3, 0.963558)
(1.4, 0.98545)
(1.5, 0.997495)
(1.6, 0.999574)
(1.7, 0.991665)
(1.8, 0.973848)
(1.9, 0.9463)
(2, 0.909297)
(2.1, 0.863209)
(2.2, 0.808496)
(2.3, 0.745705)
(2.4, 0.675463)
(2.5, 0.598472)
(2.6, 0.515501)
(2.7, 0.42738)
(2.8, 0.334988)
(2.9, 0.239249)
(3, 0.14112)
(3.1, 0.0415807)
(3.2, -0.0583741)
(3.3, -0.157746)
(3.4, -0.255541)
(3.5, -0.350783)
(3.6, -0.44252)
(3.7, -0.529836)
(3.8, -0.611858)
(3.9, -0.687766)
(4, -0.756802)
(4.1, -0.818277)
(4.2, -0.871576)
(4.3, -0.916166)
(4.4, -0.951602)
(4.5, -0.97753)
(4.6, -0.993691)
(4.7, -0.999923)
(4.8, -0.996165)
(4.9, -0.982453)
(5, -0.958924)
(5.1, -0.925815)
(5.2, -0.883455)
(5.3, -0.832267)
(5.4, -0.772764)
(5.5, -0.70554)
(5.6, -0.631267)
(5.7, -0.550686)
(5.8, -0.464602)
(5.9, -0.373877)
(6, -0.279415)
(6.1, -0.182163)
(6.2, -0.0830894)
(6.3, 0.0168139)
(6.4, 0.116549)
(6.5, 0.21512)
(6.6, 0.311541)
(6.7, 0.40485)
(6.8, 0.494113)
(6.9, 0.57844)
(7, 0.656987)
(7.1, 0.728969)
(7.2, 0.793668)
(7.3, 0.850437)
(7.4, 0.898708)
(7.5, 0.938)
(7.6, 0.96792)
(7.7, 0.988168)
(7.8, 0.998543)
(7.9, 0.998941)
(8, 0.989358)
(8.1, 0.96989)
(8.2, 0.940731)
(8.3, 0.902172)
(8.4, 0.854599)
(8.5, 0.798487)
(8.6, 0.734397)
(8.7, 0.662969)
(8.8, 0.584917)
(8.9, 0.501021)
(9, 0.412118)
(9.1, 0.319098)
(9.2, 0.22289)
(9.3, 0.124454)
(9.4, 0.0247754)
(9.5, -0.0751511)
(9.6, -0.174327)
(9.7, -0.271761)
(9.8, -0.366479)
(9.9, -0.457536)
(10, -0.544021)
};
\end{axis}
\end{tikzpicture}
\end{center}

\begin{flushleft}
{\bf Let's find deprivative for this expression:}
\end{flushleft}
$$sinx$$
The show must go on
$$x'$$
\begin{flushleft}
{\bf So, I'll try to simplify it, let's believe in my success}
\end{flushleft}
$$1 \cdot cosx$$
The show must go on
$$cosx$$
\begin{flushleft}
{\bf Final deprivative is:}
\end{flushleft}
$$cosx$$
\begin{flushleft}
{\bf Let's see how does it look like:}
\end{flushleft}
\begin{center}\begin{tikzpicture}
\begin{axis}
\addplot coordinates {
(1, 0.540302)
(1.1, 0.453596)
(1.2, 0.362358)
(1.3, 0.267499)
(1.4, 0.169967)
(1.5, 0.0707372)
(1.6, -0.0291995)
(1.7, -0.128844)
(1.8, -0.227202)
(1.9, -0.32329)
(2, -0.416147)
(2.1, -0.504846)
(2.2, -0.588501)
(2.3, -0.666276)
(2.4, -0.737394)
(2.5, -0.801144)
(2.6, -0.856889)
(2.7, -0.904072)
(2.8, -0.942222)
(2.9, -0.970958)
(3, -0.989992)
(3.1, -0.999135)
(3.2, -0.998295)
(3.3, -0.98748)
(3.4, -0.966798)
(3.5, -0.936457)
(3.6, -0.896758)
(3.7, -0.8481)
(3.8, -0.790968)
(3.9, -0.725932)
(4, -0.653644)
(4.1, -0.574824)
(4.2, -0.490261)
(4.3, -0.400799)
(4.4, -0.307333)
(4.5, -0.210796)
(4.6, -0.112153)
(4.7, -0.0123887)
(4.8, 0.087499)
(4.9, 0.186512)
(5, 0.283662)
(5.1, 0.377978)
(5.2, 0.468517)
(5.3, 0.554374)
(5.4, 0.634693)
(5.5, 0.70867)
(5.6, 0.775566)
(5.7, 0.834713)
(5.8, 0.88552)
(5.9, 0.927478)
(6, 0.96017)
(6.1, 0.983268)
(6.2, 0.996542)
(6.3, 0.999859)
(6.4, 0.993185)
(6.5, 0.976588)
(6.6, 0.950233)
(6.7, 0.914383)
(6.8, 0.869397)
(6.9, 0.815725)
(7, 0.753902)
(7.1, 0.684547)
(7.2, 0.608351)
(7.3, 0.526078)
(7.4, 0.438547)
(7.5, 0.346635)
(7.6, 0.25126)
(7.7, 0.153374)
(7.8, 0.0539554)
(7.9, -0.0460021)
(8, -0.1455)
(8.1, -0.243544)
(8.2, -0.339155)
(8.3, -0.431377)
(8.4, -0.519289)
(8.5, -0.602012)
(8.6, -0.67872)
(8.7, -0.748647)
(8.8, -0.811093)
(8.9, -0.865435)
(9, -0.91113)
(9.1, -0.947722)
(9.2, -0.974844)
(9.3, -0.992225)
(9.4, -0.999693)
(9.5, -0.997172)
(9.6, -0.984688)
(9.7, -0.962365)
(9.8, -0.930426)
(9.9, -0.889191)
(10, -0.839072)
};
\end{axis}
\end{tikzpicture}
\end{center}

{\bf My references:}
\begin{itemize}

\item Field for experiments

 https://github.com/s-a-v-a-n-n-a/Differentiator

\item Lections by Redkozubov V.V. 

 https://www.youtube.com/playlist?list=PLthfp5exSWEoItZUXCG3Bhrn3AFzw8AKK

\end{itemize}

\end{document}
